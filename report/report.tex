\documentclass{article}
\usepackage{verbatim}
\usepackage{fancyvrb}
\usepackage{colortbl}
\usepackage{caption}
\usepackage{subcaption}
\usepackage{multicol}
\usepackage{amsfonts}
\usepackage{mathtools}
\usepackage{changepage}
\usepackage{enumitem}
\usepackage{gensymb}
\usepackage{csquotes}
\usepackage{float}
\usepackage{listings}
\usepackage{amssymb}
\usepackage{amsthm}
\usepackage{amsmath}
\usepackage[hidelinks]{hyperref}
\usepackage[
	left=1in,
	right=1in,
	top=1in,
	bottom=1in
]{geometry}
\usepackage[utf8]{inputenc}
\usepackage{csquotes}
\usepackage{graphicx}
\usepackage[english]{babel}
\usepackage[
	backend=bibtex,
	style=authoryear,
	citestyle=authoryear
]{biblatex}
\addbibresource{~/.config/assets/LaTeX/auni.bib}

\setboolean{@twoside}{false}
\setlength{\parskip}{1em}
\setlength{\parindent}{4em}
\renewcommand{\baselinestretch}{1}
%\makeatletter
%\renewcommand{\@seccntformat}[1]{}
%\makeatother

\theoremstyle{definition}
\newtheorem{exmp}{Example}[section]

\author{Chris Sobczak}
\title{Survey of Post-Secondary Institution Server Software}

\begin{document}
\begin{titlepage}
\maketitle

\vspace*{\fill}

\section{Abstract}
The goal of the survey is to determine the market share
of that linux and other open source operating systems
occupy in the academic setting. So the
figure of interest is the proportion of machines that
universities run for thier websites, mail servers and other
digital services that run an open source operating system.

\end{titlepage}


\begin{flushleft}

\section{Background}
Universities spend millions of dollars each year on licenses
for their students' software offerings and also for their own
infrastructure. Increasing costs for post-secondary education
are making it less excessible and a very obvious way to reduce
costs is find free and open source alternatives to those expensive
licenses.
\textbf{(cite some study showing costs and how much licenses contribute
to the overall cost of attendance).}

In addition to the problamatic cost of closed source, proprietary
software, it is also less secure than the open source alternatives
\textbf{(cite study about this or reference the OpenBSD claims etc).}
Especially at public institutions, all of the software that staff and
students are required to interact with should also be open source
so for the sake of privacy and security, universities should not
be given the exclusive access to a specific class of student.

So in this study, I would like to establish a base-line understanding
of the proportion of free and open source servers being used by
universities around the world. \textbf{(look at poststratification)} Are there specific countries whose universities
use more open source servers than others? What kinds of services
are associated with certain operating systems?

\section{Methods}
This survey was conducted using cluster sampling.
The primary sampling unit (psu) being the name
of an institution, and the primary unit of interest
being the associated domain(s). Each of the schools
highest level registered domain names are probed to
extract all their associated subdomains, defining the secondary
sampling units (ssus).
A census was then taken of these ssus
to determine the software run at each domain name and
calculate to proportion of the school's software was open
source.

To provide an example, if Simon Fraser University (SFU)
was draw into our sample, the unit of interest would be
their registered domain name \texttt{sfu.ca}. Using the
tool \texttt{findomain}, all of the subdomains under
\texttt{sfu.ca} are collected into a file for probing.
Some example subdomains are \texttt{mailgate.sfu.ca},
\texttt{imapnew.sfu.ca}, and \texttt{canvas.sfu.ca} for
the schools email services and canvas portal. This is
only a very short list of some of the possible subdomains
as it is common to have thousands of subdomains associated
to the same top name.
Details on the tools I used in \autoref{sec:tools} \hyperref[sec:tools]{Tools}.

\subsection{Survey Design}

Using cluster
sampling, an SRS is taken of all the schools in the sampling
frame (\cite{Hipo}). This sampling frame is an open source dataset
of most of the universities in the world. Inevitably
this list will have omitted some schools, but the set
contains 9693 schools which is slightly on the low
end of estimates, which may introduce bias in the
conclusions of this study.

\subsection{Sample Size Selection}
The goal of this study is
to estimate the proportion of university servers
running an open source operating system using a 95\% confidence
interval with a margin of error of 0.03.
Therefore, using
\cite{lohr2019},
``\dots surveys in which one of the main responses of interest
is a proportion, it is often easiest to use that response
in setting the sample size.
For large populations, $S^2 \approx p(1-p)$, which
attains its maximal value when $p=1/2$. So using
$n_0=1.96^2/(4e^2)$ will result in a 95\% CI with width at most
$2e$.''

$$
	n_0
	=
	\frac{
		z^2_{\alpha/2}S^2
	}{
		e^2
	}
	=
	\frac{
		1.96^2(\frac{1}{2})(1-\frac{1}{2})
	}{
		e^2
	}
	\approx
	1067
$$

No need to use the finite population correction adjustment
since the sample size
is reasonable compared to the population size and the full
dataset can be collected with a reasonable amount of resources.

\subsection{Taking the Sample}
With an appropriate sample size of $n=1067$,
using an R script to
draw the sample, the selected school domains
are saved in the \texttt{data.Rda} file found
in the GitHub repository.
\textbf{\underline{The sampling script
can be found in the
appendix}}.
Within the sample, 1049 schools have only
one registered domain, 16 schools have two
registered domains and 2 have three
domains.

$$
	N=9693, \ n=1067, \ m_0=2539008
$$

The following 18 schools that were drawn in the sample
have more than one domain name registered:
Augusta University,
University of Manchester,
Universidad del País Vasco,
Chinju National University of Education,
Royal Holloway and Bedford New College,
Northeastern University,
University of the Pacific,
Kwangju University,
Kwangwoon University,
University of Massachusetts at Lowell,
St. Mary's University,
University of Essex,
Chonnam National University,
Hanshin University,
Savannah College of Art and Design,
University of Technology Sydney,
Universitat Pompeu Fabra, and
Kyungil University.

These 16 schools with two domains and 2 schools with three domains
accounts for the total 1087 ($1067 + (16 \textrm{ duplicates}) +
(2\times2 \textrm{ more duplicates}) = 1087$) domains in the
\texttt{subdomains/} directory of the GitHub repository.

All domains were separated onto their own line of the \texttt{probing/domains}
file and processed with \texttt{findomain}.
When extracting the subdomains, the \texttt{gen-subdomains}
script only processed 1079 domains, identifying an inconsistency.
Three of these domains that were not processed were \texttt{aloma.edu},
\texttt{student.uts.edu.au}, and \texttt{www.clcmn.edu}. In the case of
\texttt{aloma.edu}, this is just a typo in the sampling frame for the
Alamo Colleges' domain, which naturally is corrected to \texttt{alamo.edu}.
The next missing domain \texttt{student.uts.edu.au}, for the University
of Technology Sydney in Australia which is just a subdomain for their
website \texttt{uts.edu.au}, identified as a duplicate domain.
Finally, \texttt{www.clcmn.edu} for
Central Lakes College-Brainerd is another subdomain for their college
that was already extracted from \texttt{clcmn.edu}, another duplicated domain.

In the github repository, the file \texttt{undup-domains} is the audited file containing all
of the highest level domains for the sample.

The next step will be ensuring that two related domains did not go through
the process, for example making sure that we did not run \texttt{gen-domains}
on \texttt{sfu.ca} and \texttt{mail.sfu.ca}, since this would likely result in
duplicated information.

Using the line \texttt{cat subdomain/* | uniq -d} we can see that
there were no repeated lines.




\section{Results}
The full results dataset of proportion of servers at
each psu is available at the \textbf{github names \dots}.


\subsection{Discussion and Conclusion}
\textbf{Any anticpiated shortcomings in your study desgin
and their impact on your conclusions about the questions
of interest.}
\begin{itemize}
\item Non-responses
\item Shortcomings in the sampling frame (may not actually contain `all' schools in the world)
\end{itemize}



\section{Tools} \label{sec:tools}
For taking the raw json file and selecting the sample, I used R and the \texttt{rjson}
package. All project source files can be found at this
\href{https://github.com/chrissobczak/os-survey}{GitHub} repository.
The R script outputs the base second and third level subdomains
into a file with one domain per line, for which I run \texttt{findomain -t}.
This tool takes a domain and searches various databases and tests the domain
for subdomains associated with it. I take this and output all the subdomains
into a file for each school, and then test if the service is up.

Documentation for the tools can be found in the Appendix
\subsection{Extracting Info From SSUs}
\texttt{curl -I}


\end{flushleft}
\printbibliography
\section{Appendix}
\end{document}
