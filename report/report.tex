\documentclass{article}
\usepackage{verbatim}
\usepackage{fancyvrb}
\usepackage{colortbl}
\usepackage{caption}
\usepackage{subcaption}
\usepackage{multicol}
\usepackage{amsfonts}
\usepackage{mathtools}
\usepackage{changepage}
\usepackage{enumitem}
\usepackage{gensymb}
\usepackage{csquotes}
\usepackage{float}
\usepackage{listings}
\usepackage{amssymb}
\usepackage{amsthm}
\usepackage{amsmath}
\usepackage[hidelinks]{hyperref}
\usepackage[
	left=1in,
	right=1in,
	top=1in,
	bottom=1in
]{geometry}
\usepackage[utf8]{inputenc}
\usepackage{csquotes}
\usepackage{graphicx}
\usepackage[english]{babel}
\usepackage[
	backend=bibtex,
	style=authoryear,
	citestyle=authoryear
]{biblatex}
\addbibresource{~/.config/assets/LaTeX/auni.bib}

\setboolean{@twoside}{false}
\setlength{\parskip}{1em}
\setlength{\parindent}{4em}
\renewcommand{\baselinestretch}{1}
%\makeatletter
%\renewcommand{\@seccntformat}[1]{}
%\makeatother

\theoremstyle{definition}
\newtheorem{exmp}{Example}[section]

\author{Chris Sobczak}
\title{Survey of Operating Systems Used by World Universities}

\begin{document}
\maketitle
\begin{flushleft}



\section{Introduction}
The goal of the survey is to determine the market share
of that linux and other open source operating systems
occupy in the academic setting. So the
figure of interest is the proportion of machines that
universities run for thier websites, mail servers and other
digital services that run an open source operating system.

\section{Methodologies}
I have used a dataset of all known domains registered by
universities around the world (\cite{Hipo}). I have taken a simple random
sample of these schools and then extracted all of the corresponding
subdomains.

For example, Simon Fraser University (SFU) has \url{sfu.ca} registered with
the Canadian Internet Registration Authority (CIRA) and
they have thousands of subdomains under \url{sfu.ca} such as \url{mail.sfu.ca},
\url{canvas.sfu.ca} and \url{go.sfu.ca}. A lot of the subdomains do not
contain relevant web content or are otherwise unused, so to filter out these
subdomains before trying to identify the operating systems being
used to run the surver at that address, I have used the \texttt{host} DNS
lookup tool and other methods for checking if there are actual services running
at that address. More about the tools I used in \autoref{sec:tools} \hyperref[sec:tools]{Tools}.

\subsection{Sampling Frame}


\subsection{Sample Size Selection}
$e=0.03$ and $\alpha=0.05$ \cite{lohr2019}

We want to estimate the proportion of university servers
run an open source operating system using a 95\% confidence
interval with a margin of error of 0.03.

\cite{lohr2019} page 47:
``In surveys in which one of the main responses of interest
is a proportion, it is often easiest to use that response
in setting the sample size.
For large populations, $S^2 \approx p(1-p)$, which
attains its maximal value when $p=1/2$. So using
$n_0=1.96^2/(4e^2)$ will result in a 95\% CI with width at most
$2e$.''

$$
	n_0
	=
	\frac{
		z^2_{\alpha/2}S^2
	}{
		e^2
	}
	=
	\frac{
		1.96^2(\frac{1}{2})(1-\frac{1}{2})
	}{
		e^2
	}
	\approx
	1067
$$

No need to use the fpc adjustment since the sample size
is reasonable compared to the population size.

\section{Tools} \label{sec:tools}


\end{flushleft}
\printbibliography
\end{document}
